\documentclass[../Main.tex]{subfiles}

\usepackage{amsfonts}

\begin{document}
\chapter{Vector Spaces}

\section{$\mathbb{R}^{n}$ and $\mathbb{C}^{n}$}

We skip this section.

\section{Difinition of Vector Space}

\thmp{(1B-1)}{
Prove that $-(-v) = v$ for every $v \in V$.
}{
For $v \in V$, we have
\begin{equation}
    -(-v) = -(-v) + (-v) + v = v.
\end{equation}
Thus we know the additive inverse of the additive inverse of $v$ is itself.
}

\thmp{(1B-2)}{
\label{2.2}
Suppose $a \in \textbf{F}$, $v \in V$, and $av = 0$. Prove that $a = 0$ or $v = 0$.
}{
If $a = 0$, then we are done.\\
If $a \neq 0$, then
\begin{equation}
    v = (\frac{1}{a} \cdot a)v = \frac{1}{a} (av) = 0.
\end{equation}
}

\thmp{(1B-3)}{
Suppose $v, w \in V$. Explain why there exists a unique $x \in V$ such that $v + 3x = w$.
}{
Let $x = \frac{w - v}{3}$, then
\begin{equation}
    v + 3x = w.
\end{equation}
This show existence. Now we show the uniqueness.
Suppose there is an $x'$ that satisfies $v + 3x' = w$, then
\begin{equation}
    3(x - x') = 3x - 3x' = (w - v) - (w - v) = 0.
\end{equation}
By Exercise 2.\ref{2.2}, we must have $x - x' = 0$, thus $x = x'$.
}

\thmp{(1B-4)}{
    The empty set is not a vector space.
    The empty set fails to satisfy only one of the requirements listed in the definition of a vector space (1.20). 
    Which one?
}{
    Additive identity.
    In an empty set $\emptyset$, there does not exist an element 0 that $v + 0 = v$ for all $v \in \emptyset$.
}

\rmk{
    Exercise 2.4 shows that the additive identity condition can be replaced with the condition that the set is not empty(because then taking $u \in U$ and multiplying it by 0 would imply that $0 \in U$).
}

\thmp{(1B-6)}{
Let $\infty$ and $-\infty$ denote two distinct objects, neither of which is in \textbf{R}.
Define an addition and scalar multiplication on $\textbf{R} \cup \{\infty , -\infty\}$ as you could guess from the notation.
Specifically, the sum and product of two real numbers is as usual, and for $t \in \textbf{R}$ define
\begin{equation*}
    t\infty =
    \begin{cases}
        -\infty& if t < 0,\\
        0& if t = 0,\\
        \infty& if t > 0,
    \end{cases}
    t(-\infty) = 
    \begin{cases}
        \infty& if t < 0,\\
        0& if t = 0,\\
        -\infty& if t > 0,
    \end{cases}
\end{equation*}
and
\begin{gather*}
    t + \infty = \infty + t = \infty + \infty = \infty,\\
    t + (-\infty) = (-\infty) + t = (-\infty) + (-\infty) = -\infty,\\
    \infty + (-\infty) = (-\infty) + \infty = 0.
\end{gather*}
With these operations of addition and scalar multiplication, is $\textbf{R} \cup \{\infty , -\infty\}$ a vector space over \textbf{R}?
Explain.
}{
We can notice that
\begin{equation}
    \infty = (2 + (-1))\infty = 2\infty + (-1)\infty = \infty + (-\infty) = 0.
\end{equation}
For $\infty \neq 0$, the set doesn't follow the distributive property.
Thus $\textbf{R} \cup \{\infty , -\infty\}$ is not a vector space.
}

\thmp{(1B-8)}{
Suppose $V$ is a real vector space.
\begin{itemize}
    \item The \textit{complexification} of $V$, denoted by $V_{\textbf{C}}$, equals $V \times V$. An element of $V_{\textbf{C}}$ is an ordered pair $(u, v)$, where $u, v \in V$, but we write this as $u + iv$.
    \item Addition on VC is defined by 
    \begin{equation*}
        (u_1 + iv_1) + (u_2 + iv_2) = (u_1 + u_2) + i(v_1 + v_2)
    \end{equation*}
    for all $u_1, v_1, u_2, v_2 \in V$.
    \item Complex scalar multiplication on $V_{\textbf{C}}$ is defined by 
    \begin{equation*}
        (a + bi)(u + iv) = (au - bv) + i(av + bu)
    \end{equation*}
    for all $a, b \in \textbf{R}$ and all $u, v \in V$.
\end{itemize}
Prove that with the definitions of addition and scalar multiplication as above, $V_{\textbf{C}}$ is a complex vector space.
}{
Just verify the six properties of vector spaces.
For example:

\textbf{commutativity}
\begin{equation}
    \begin{aligned}
        (u_1 + iv_1) + (u_2 + iv_2) =& (u_1 + u_2) + i(v_1 + v_2)\\
        =& (u_2 + u_1) + i(v_2 + v_1)\\
        =& (u_2 + iv_2) + (u_1 + iv_1)
    \end{aligned}
\end{equation}
for all $u_1, u_2, v_1, v_2 \in V$.
The remaining five properties are the same.
Thus we have the complex vector space $V_{\textbf{C}}$.
}

\section{Subspaces}

\thmp{(1C-5)}{
Is $\textbf{R}^2$ a subspace of the complex vector space $\textbf{C}^2$?
}{
Notice that subspaces of $\textbf{C}^2$ are closed under scalar multiplication in \textbf{C}, then
\begin{equation*}
    i(1, 1) = (i, i) \notin \textbf{R}^2.
\end{equation*}
Thus $\textbf{R}^2$ is not a subspace of $\textbf{C}^2$.
}

\thmp{(1C-6)}{
\begin{enumerate}
    \item[(a)] Is $\{(a, b, c) \in \textbf{R}^3 : a^3 = b^3\}$ a subspace of $\textbf{R}^3$?
    \item[(b)] Is $\{(a, b, c) \in \textbf{C}^3 : a^3 = b^3\}$ a subspace of $\textbf{C}^3$?
\end{enumerate}
}{
(a)\\
The equation $a^3 = b^3$ has the only solution $a = b$ in $\textbf{R}$, hence
\begin{equation}
    \{(a, b, c) \in \textbf{R}^3 : a^3 = b^3\} = \{(a, b, c) \in \textbf{R}^3 : a = b\}
\end{equation}
is obviously a subspace of $\textbf{R}^3$.\\
(b)\\
In $\textbf{C}^3$, we have
\begin{equation}
    \Bigg(1, \frac{-1 + \sqrt{3}i}{2}, 0\Bigg) \in \{(a, b, c) \in \textbf{C}^3 : a^3 = b^3\}
\end{equation}
and
\begin{equation}
    \Bigg(1, \frac{-1 - \sqrt{3}i}{2}, 0\Bigg) \in \{(a, b, c) \in \textbf{C}^3 : a^3 = b^3\}.
\end{equation}
However,
\begin{equation*}
    \Bigg(1, \frac{-1 + \sqrt{3}i}{2}, 0\Bigg) + \Bigg(1, \frac{-1 - \sqrt{3}i}{2}, 0\Bigg) = (2, -1, 0) \notin \{(a, b, c) \in \textbf{C}^3 : a^3 = b^3\}.
\end{equation*}
Hence $\{(a, b, c) \in \textbf{C}^3 : a^3 = b^3\}$ is not a vector space.
}







\end{document}
