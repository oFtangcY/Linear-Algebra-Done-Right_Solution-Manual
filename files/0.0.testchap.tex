\documentclass[../Main.tex]{subfiles}

\usepackage{amsfonts}
\usepackage{fdsymbol}

\begin{document}
\chapter{Vector Spaces}

\section{$\mathbb{R}^{n}$ and $\mathbb{C}^{n}$}

We skip this section.

\section{Difinition of Vector Space}

\thmp{(1B-1)}{
Prove that $-(-v) = v$ for every $v \in V$.
}{
For $v \in V$, we have
\begin{equation}
    -(-v) = -(-v) + (-v) + v = v.
\end{equation}
Thus we know the additive inverse of the additive inverse of $v$ is itself.
}

\thmp{(1B-2)}{
\label{2.2}
Suppose $a \in \textbf{F}$, $v \in V$, and $av = 0$. Prove that $a = 0$ or $v = 0$.
}{
If $a = 0$, then we are done.\\
If $a \neq 0$, then
\begin{equation}
    v = (\frac{1}{a} \cdot a)v = \frac{1}{a} (av) = 0.
\end{equation}
}

\thmp{(1B-3)}{
Suppose $v, w \in V$. Explain why there exists a unique $x \in V$ such that $v + 3x = w$.
}{
Let $x = \frac{w - v}{3}$, then
\begin{equation}
    v + 3x = w.
\end{equation}
This show existence. Now we show the uniqueness.
Suppose there is an $x'$ that satisfies $v + 3x' = w$, then
\begin{equation}
    3(x - x') = 3x - 3x' = (w - v) - (w - v) = 0.
\end{equation}
By Exercise 2.\ref{2.2}, we must have $x - x' = 0$, thus $x = x'$.
}

\thmp{(1B-4)}{
    The empty set is not a vector space.
    The empty set fails to satisfy only one of the requirements listed in the definition of a vector space (1.20). 
    Which one?
}{
    Additive identity.
    In an empty set $\varnothing$, there does not exist an element 0 that $v + 0 = v$ for all $v \in \varnothing$.
}

\rmk{
    Exercise 2.4 shows that the additive identity condition can be replaced with the condition that the set is not empty(because then taking $u \in U$ and multiplying it by 0 would imply that $0 \in U$).
}

\thmp{(1B-6)}{
Let $\infty$ and $-\infty$ denote two distinct objects, neither of which is in \textbf{R}.
Define an addition and scalar multiplication on $\textbf{R} \cup \{\infty , -\infty\}$ as you could guess from the notation.
Specifically, the sum and product of two real numbers is as usual, and for $t \in \textbf{R}$ define
\begin{equation*}
    t\infty =
    \begin{cases}
        -\infty& \text{if}\; t < 0,\\
        0& \text{if}\; t = 0,\\
        \infty& \text{if}\; t > 0,
    \end{cases}
    \qquad
    t(-\infty) = 
    \begin{cases}
        \infty& \text{if}\; t < 0,\\
        0& \text{if}\; t = 0,\\
        -\infty& \text{if}\; t > 0,
    \end{cases}
\end{equation*}
and
\begin{gather*}
    t + \infty = \infty + t = \infty + \infty = \infty,\\
    t + (-\infty) = (-\infty) + t = (-\infty) + (-\infty) = -\infty,\\
    \infty + (-\infty) = (-\infty) + \infty = 0.
\end{gather*}
With these operations of addition and scalar multiplication, is $\textbf{R} \cup \{\infty , -\infty\}$ a vector space over \textbf{R}?
Explain.
}{
We can notice that
\begin{equation}
    \infty = (2 + (-1))\infty = 2\infty + (-1)\infty = \infty + (-\infty) = 0.
\end{equation}
For $\infty \neq 0$, the set doesn't follow the distributive property.
Thus $\textbf{R} \cup \{\infty , -\infty\}$ is not a vector space.
}

\thmp{(1B-8)}{
Suppose $V$ is a real vector space.
\begin{itemize}
    \item The \textit{complexification} of $V$, denoted by $V_{\textbf{C}}$, equals $V \times V$. An element of $V_{\textbf{C}}$ is an ordered pair $(u, v)$, where $u, v \in V$, but we write this as $u + iv$.
    \item Addition on VC is defined by 
    \begin{equation*}
        (u_1 + iv_1) + (u_2 + iv_2) = (u_1 + u_2) + i(v_1 + v_2)
    \end{equation*}
    for all $u_1, v_1, u_2, v_2 \in V$.
    \item Complex scalar multiplication on $V_{\textbf{C}}$ is defined by 
    \begin{equation*}
        (a + bi)(u + iv) = (au - bv) + i(av + bu)
    \end{equation*}
    for all $a, b \in \textbf{R}$ and all $u, v \in V$.
\end{itemize}
Prove that with the definitions of addition and scalar multiplication as above, $V_{\textbf{C}}$ is a complex vector space.
}{
Just verify the six properties of vector spaces.
For example:

\textbf{commutativity}
\begin{equation}
    \begin{aligned}
        (u_1 + iv_1) + (u_2 + iv_2) =& (u_1 + u_2) + i(v_1 + v_2)\\
        =& (u_2 + u_1) + i(v_2 + v_1)\\
        =& (u_2 + iv_2) + (u_1 + iv_1)
    \end{aligned}
\end{equation}
for all $u_1, u_2, v_1, v_2 \in V$.
The remaining five properties are the same.
Thus we have the complex vector space $V_{\textbf{C}}$.
}

\section{Subspaces}

\thmp{(1C-5)}{
Is $\textbf{R}^2$ a subspace of the complex vector space $\textbf{C}^2$?
}{
Notice that subspaces of $\textbf{C}^2$ are closed under scalar multiplication in \textbf{C}, then
\begin{equation*}
    i(1, 1) = (i, i) \notin \textbf{R}^2.
\end{equation*}
Thus $\textbf{R}^2$ is not a subspace of $\textbf{C}^2$.
}

\thmp{(1C-6)}{
\begin{enumerate}
    \item[(a)] Is $\{(a, b, c) \in \textbf{R}^3 : a^3 = b^3\}$ a subspace of $\textbf{R}^3$?
    \item[(b)] Is $\{(a, b, c) \in \textbf{C}^3 : a^3 = b^3\}$ a subspace of $\textbf{C}^3$?
\end{enumerate}
}{
(a)\\
The equation $a^3 = b^3$ has the only solution $a = b$ in $\textbf{R}$, hence
\begin{equation}
    \{(a, b, c) \in \textbf{R}^3 : a^3 = b^3\} = \{(a, b, c) \in \textbf{R}^3 : a = b\}
\end{equation}
is obviously a subspace of $\textbf{R}^3$.\\
(b)\\
In $\textbf{C}^3$, we have
\begin{equation}
    \Bigg(1, \frac{-1 + \sqrt{3}i}{2}, 0\Bigg) \in \{(a, b, c) \in \textbf{C}^3 : a^3 = b^3\}
\end{equation}
and
\begin{equation}
    \Bigg(1, \frac{-1 - \sqrt{3}i}{2}, 0\Bigg) \in \{(a, b, c) \in \textbf{C}^3 : a^3 = b^3\}.
\end{equation}
However,
\begin{equation*}
    \Bigg(1, \frac{-1 + \sqrt{3}i}{2}, 0\Bigg) + \Bigg(1, \frac{-1 - \sqrt{3}i}{2}, 0\Bigg) = (2, -1, 0) \notin \{(a, b, c) \in \textbf{C}^3 : a^3 = b^3\}.
\end{equation*}
Hence $\{(a, b, c) \in \textbf{C}^3 : a^3 = b^3\}$ is not a vector space.
}

\thmp{(1C-11)}{
    Prove that the intersection of every collection of subspaces of $V$ is a subspace of $V$.
}{
    Assume $U_i$ is a subspace of $V$ for $i \in I$, then prove $\cap_{i \in I}U_i$ is a subspace of $V$.

    \textbf{Additive identity}
    \begin{equation}
        0 \in U_i \quad \forall i \in I \Rightarrow 0 \in \cap_{i \in I}U_i.
    \end{equation}

    \textbf{Closed under addition} For all $u, v \in \cap_{i \in I}U_i$, $u, v \in U_i ,\; \forall i \in I$, thus
    \begin{equation}
        u + v \in U_i ,\; \forall i \in I \Rightarrow u + v \in \cap_{i \in I}U_i
    \end{equation}
    which shows $\cap_{i \in I}U_i$ is closed under addition.

    \textbf{Closed under scalar multiplication} For all $w \in \cap_{i \in I}U_i$, $w \in U_i ,\; \forall i \in I$, thus
    \begin{equation}
        \lambda w \in U_i ,\; \forall i \in I \Rightarrow \lambda w \in \cap_{i \in I}U_i
    \end{equation}
    which shows $\cap_{i \in I}U_i$ is closed under scalar multiplication.
}

\thmp{(1C-12)}{
    Prove that the union of two subspaces of $V$ is a subspace of $V$ if and only if one of the subspaces is contained in the other.
}{
    Let $U_1, U_2$ be subspaces of $V$.
    Suppose $u \in U_1$ and $u \notin U_2$ while $v \in U_2$ and $v \notin U_1$, then we have $u, v \in U_1 \cup U_2$ but $u + v \notin U_1$ or $U_2$, thus not in $U_1 \cup U_2$.
    Hence $U_1 \cup U_2$ is not closed under addition, and we get a contradiction.
    So either $\forall u \in U_1, \; u \in U_2$ or $\forall v \in U_2, \; v \in U_1$, or $U_1 = U_2$.
}

\thmp{(1C-13)}{
    Prove that the union of three subspaces of $V$ is a subspace of $V$ if and only if one of the subspaces contains the other two.
}{
    Let $U_1, U_2, U_3$ be subspaces of $V$, consider $u \in U_1$ and $v \in U_2$.
    If $U \cup V \neq U$ or $V$, then we can assume there exist $u \notin U_2$ and $v \notin U_1$.
    From the assumption we know $u + v \in \cup_{i = 1}^3 U_i$ and $u + v$ not in $U_1$ or $U_2$, thus $u + v \in U_3$.
    For the same reason we know $u + 2v$ and $2u + v$ are in $U_3$, and because $U_3$ is closed under addition and additive inverse, we have $u, v \in U_3$.
    Hence $\complement_{U_1}U_1 \cap U_2$ and $\complement_{U_2}U_1 \cap U_2$ are contained by $U_3$.

    For $w \in U_1 \cap U_2$, taking $u \in \complement_{U_1}U_1 \cap U_2$, then $u + w$ in $U_1$ but not in $U_2$, and we come back to the above condition.
    Thus we can prove $U_1 \cap U_2 \subset U_3$, which means $U_3$ contains $U_1$ and $U_2$.

    Now consider the consition that $U_1 \subset U_2$.
    If $U_2 \subset U_3$, then we are done.
    If $U_1 \subset U_3$, then consider $u \in \complement_{U_2}U_1$ and $v \in \complement_{U_3}U_1$, because the union of $U_1, U_2, U_3$ is closed under addition, $u + v \in \cup_{i = 1}^3 U_i$.
    Evidently $u + v \notin U_1$, if $u + v \in U_2$, then we can prove $U_2 = U_3$;
    the other side is the same.

    From the conditions above, we prove the proposition in the question.
}

\rmkb{
    Problem 15-19 in Exercises 1C prove some properties of the addition in the set of subspaces of $V$, including additive identity, commutativity, associativity.
    From these propositions we know the group of subspaces of $V$ is not an abel group.
}

\thmp{(1C-21)}{
    Suppose
    \begin{equation*}
        U = \{(x, y, x + y, x - y, 2x) \in \textbf{F}^5 : x, y \in \textbf{F}\}.
    \end{equation*}
    Find a subspace $W$ of $\textbf{F}^5$ such that $\textbf{F}^5 = U \oplus W$.
}{
    Let $W = \{(0, 0, u, v, w) \in \textbf{F}^5 : u, v, w \in \textbf{F}\}$.
    Consider $(x, y, x + y, x - y, 2x) \in U \cap W$, then we have $x = y = 0$, which means $U \cap W = \{0\}$.
    Hence $\textbf{F}^5 = U \oplus W$ by 1.45.
}

\rmk{
    Problem 20-22 in Exercises 1C are the same type.
    To solve these problems, we first write a subspace $W$ that the union of $W$ and the given subspaces is the vector space asked for.
    Then prove the intersection of the subspaces is $\{0\}$.
}

\thmp{(1C-23)}{
    Prove or give a counterexample: If $V_1, V_2, U$ are subspaces of $V$ such that
    \begin{equation*}
        V = V_1 \oplus U \quad \text{and} \quad V = V_2 \oplus U,
    \end{equation*}
    then $V_1 = V_2$.
}{
    Consider the example in Exercise 3.6.
    We change $W$ into $W' = \{(0, z, 0, v, w) \in \textbf{F}^5 : u, v, w \in \textbf{F}\}$, and it is easy to verify $\textbf{F}^5 = U \oplus W'$.
    Hence we give a counterexample.
}

\thmp{(1C-24)}{
    A function $f : \textbf{R} \rightarrow \textbf{R}$ is called \textit{even} if 
    \begin{equation*}
        f (-x) = f (x)
    \end{equation*}
    for all $x \in \textbf{R}$.
    A function $f : \textbf{R} \rightarrow \textbf{R}$ is called \textit{odd} if
    \begin{equation*}
        f (-x) = -f (x)
    \end{equation*}
    for all $x \in R$.
    Let $V_e$ denote the set of real-valued even functions on \textbf{R} and let $V_o$ denote the set of real-valued odd functions on \textbf{R}.
    Show that $\textbf{R}^{\textbf{R}} = V_e \oplus V_o$.
}{
    For $h : \textbf{R} \rightarrow \textbf{R} \in \textbf{R}^{\textbf{R}}$, suppose $h(x) = f(x) + g(x)$, where $f \in V_e$ and $g \in V_o$.
    Then we must verify $h(-x) = f(-x) + g(-x) = f(x) - g(x)$.
    From the above discription, let
    \begin{equation}
        f(x) = \frac{h(x) + h(-x)}{2}
    \end{equation}
    and
    \begin{equation}
        g(x) = \frac{h(x) - h(-x)}{2}.
    \end{equation}
    Then we have $\textbf{R}^{\textbf{R}} = V_e + V_o$.

    Consider $f_0 \in V_e \cap V_o$,
    \begin{equation}
        f_0(x) = f_0(-x) = -f_0(x), \; \forall x \in \textbf{R}.
    \end{equation}
    Thus $f_0(x) \equiv 0$, $V_e \cap V_o = \{0\}$.
    By 1.45, $\textbf{R}^{\textbf{R}} = V_e \oplus V_o$.
}

\rmk{
    From Exercise 3.8 we know every real-valued function can be divided into the sum of an even function and an odd function.
}





\end{document}
